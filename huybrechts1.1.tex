\documentclass[10pt,letter]{article}
\usepackage{amsmath,amssymb,breqn,enumitem,fullpage,graphicx,setspace,mathtools,pst-node,stmaryrd,tikz-cd}
\onehalfspacing
\usepackage{fullpage}

\begin{document}
\begin{center} 
{\bf Huybrechts 1.1} \\
Holly Mandel 05/11/2018
\end{center}

\paragraph*{1.1.1} If $f$ is such a holomorphic function, then so is $g = e^{if}$. But $\vert g \vert \leq 1$ because  $\Re(if) = - \Im(f)  < 0$. Therefore $g$ is constant, so $f$ is constant.  
\paragraph*{1.1.2} 
\paragraph*{1.1.3} If $f: U \subseteq \mathbb{C}^n \rightarrow \mathbb{C}$ has a local maximum at $(p_1,...,p_n)$ in the interior of $U$, then the holomorphic function $z \mapsto f(z,p_2,...,p_n)$ has a local maximum in the interior of its domain, a contradiction. The same idea is used to prove the identity principle.
\paragraph*{1.1.4} 
\end{document}