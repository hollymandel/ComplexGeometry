\documentclass[10pt,letter]{article}
\usepackage{amsmath,amssymb,graphicx,setspace,fullpage,breqn,mathtools,stmaryrd}
\onehalfspacing
\usepackage{fullpage}
\DeclareMathOperator*{\argmin}{arg\,min}

\begin{document}
\noindent 
January 2019
\begin{center}
\textbf{Chapter 2.4: Projective Space}\\ Huybrechts, \textit{Complex Geometry}

\line(1,0){250}
\end{center}
\subparagraph{2.4.1} This is shown by induction on Proposition 2.4.7. The case $k = 1$ is given by Corollary 2.4.9. For the inductive step, we note that $M_k = Z(f_1) \cap ... \cap Z(f_k)$ is a hypersurface of degree $d_k$ of the projective manifold $M_{k-1} = Z(f_1) \cap ... \cap Z(f_{k-1})$. This is because $f_k$ gives a section of $\mathcal{O}(d_k)\vert_{Z(f_1) \cap ... \cap Z(f_{k-1})}$. Therefore
\begin{align*} K_{M_k} &\simeq (K_{M_{k-1}} \otimes \mathcal{O}(d_k)\vert_{M_{k-1}})\vert_{M_k}   \\ 
&\simeq (\mathcal{O}(\sum_{j=1}^{k-1} d_j) \vert_{M_{k-1}} \otimes \mathcal{O}(d_k)\vert_{M_{k-1}})\vert_{M_k}  \\ 
&\simeq \mathcal{O}(\sum_{j=1}^k d_j-(n+1))\vert_{M_k}.
\end{align*}
\subparagraph{2.4.2} We have the Euler sequence \[ 0 \rightarrow \mathcal{O} \rightarrow \oplus_{i=0}^n \mathcal{O}(1) \rightarrow \mathcal{T}_{\mathbb{P}^n} \rightarrow 0.\] 
The first map is given by 
\[ \lambda \mapsto (\lambda X_0,...,\lambda X_n).\]
The second map is given by
\[ (f_0(X_0,...,X_n),...,f_n(X_0,...,X_n)) \mapsto \sum_{i=0}^n f_i(X_0,...,X_n) D_i,\] where $D_i$ is the image of $\partial_i$ under the projection $\pi: \mathbb{C}^{n+1}\setminus 0 \rightarrow \mathbb{P}^n$. It is verified that at each point the kernel of $D\pi$ is the image of the first map, so the image of $(f_0(X_0,...,X_n),...,f_n(X_0,...,X_n))$ vanishes only if it is in this image. Since each $f_i$ is a homogenous polynomial, this can only occur either if the point is in the image on all of $\mathbb{C}^{n+1}$, or if all the polynomials vanish at a point. By linear algebra, this means that for any $n$ linear forms that are independent and such that $\langle f_1,...,f_n \rangle + \text{Im}$ spans, there is a vector field on $\mathbb{P}^n$ that vanishes only at the point corresponding to the line that is the kernel of the $f_i$. So such vector fields exist and each only vanishes at one point. CLEAN UP.
 
\subparagraph{2.4.3} \begin{enumerate}
\item By Corollary 2.4.9, $K_X \simeq \mathcal{O}(-1)$, which has no global sections. Therefore by the same reasoning as Corollary 2.4.6, $\text{kod}(\mathbb{P}^n) = -\infty$. 
\item In this case $K_x \simeq \mathcal{O}$. Observing the group law on $\mathcal{R}(X)$, we see that $\mathcal{R}(X) \simeq \mathbb{C}[x]$. Therefore $\mathcal{Q}(X)$ has transcendence degree $1$, so $\text{kod}(\mathbb{P}^n) = 0$.
\item See (1).
\item In this case $K_x \simeq \mathcal{O}(1)$, so $H^0(X,K_x)$ is the collection of linear forms in $n+1$ variables, so $\mathcal{R}(X) \simeq \mathbb{C}[x_0,...,x_n]$. Therefore $\text{kod}(\mathbb{P}^n) = n$.
\end{enumerate}

\subparagraph{2.4.4}
First I claim that a holomorphic $p$-form induces a holomorphic top form an a $p$-dimensional complex submanifold. For let $\omega \in \Omega^p$ and let $M \subseteq X$ be such a submanifold and let $i$ be the inclusion map. Then \[ \bar{\partial} i^{\ast} \omega = i^{\ast} \bar{\partial} \omega = 0.\]

Now I claim that for $p > 0$, if $\omega$ is nonzero, there exists a $p$-dimensional complete intersection such that $i^{\ast} \omega \neq 0$. By the identity principle, $\omega$ does not vanish on any open subset of $\mathbb{P}^n$, so we can write in the coordinates on $U_0$  ($Z_{i/0} = X_i/X_0$) \[ \omega = \sum \alpha_{i_1,...,i_p} dZ_{i_1/0} \wedge ... \wedge dZ_{i_p/0}.\]
with at least one $\alpha_{i_1,...,i_p} \neq 0$. Choosing such an $\alpha$, we let $M$ be the projective submanifold $\mathbb{P}(\langle X_0,X_{i_1},...,X_{i_p} \rangle) = Z(X_{j_1}) \cap ... \cap Z(X_{j_{n-p}})$. In the coordinates on $M \cap U_0$, \[ i^{\ast} \omega = \alpha_{i_1,...,i_p} dZ_{i_1/0} \wedge ... \wedge dZ_{i_p/0} \neq 0,\] as desired. 

This proves that $H^{0}(\mathbb{P}^n,\Omega^p) = 0$ for all $p > 0$. For by Exercise 2.4.1, $K_M \simeq \mathcal{O}((n-p)-(n+1))$, so it has no global sections. 

\subparagraph{2.4.5} I claim that there is a vector bundle isomorphism \[ \mathcal{O}_{\mathbb{P}^1} \oplus \mathcal{O}_{\mathbb{P}^1}(n) \simeq Z(x_0^n y_1 - x_1^n y_2) \subset \mathbb{P}^1 \times \mathbb{C}^3.\] It will follow that the projectivization of these two vector bundles are isomorphic. But for each $X = [x_0:x_1] \in \mathbb{P}^1$, the projectivization of the fiber over $X$ is the projective subspace given by the same polynomial. NEEDS MORE ELABORATION. ACTUALLY GIVE AN ISO. 

To prove the isomorphism suffices to show that these bundles have the same transition functions. Since $Z_0^n$ defines a global section of $\mathcal{O}_{\mathbb{P}^1}(n)$, we can compute the transition functions as the ratio of the coordinate representations of this section (Prop 2.3.8). With respect to the standard charts $(U_i)$, we have that \[ \rho_{ij} = (Z_0^n/Z_i^n)/(Z_0^n/Z_j^n) = Z_j^n/Z_i^n.\] Therefore the transition functions on  $\mathcal{O}_{\mathbb{P}^1} \oplus \mathcal{O}_{\mathbb{P}^1}(n)$ are 
\[ r_{ij} = 1 \oplus (Z_j/Z_i)^n.\]  WHAT COORDINATES IS THIS IN?

On the other hand, over the chart $U_0 \subset \mathbb{P}^1$, the hypersurface is parameterized by
\[ (x,s,t) \mapsto ((1,x),(s,tx^n,t)),\]
while over $U_1$ it is parameterized by
\[ (y,u,v) \mapsto ((y,1),(y,v,y^nv)).\]
Thus $u(x,s,t) = s$ and $v(x,s,t) = tx^n$, so
\[ r'_{01} = 1 \oplus (Z_1/Z_0)^n.\] 
This proves that the two bundles have the same transition functions. 

\subparagraph{2.4.6} The tangent and contangent bundle of a direct product is the direct sum of the tangent and cotangent bundles. 

$K(\mathbb{P}^n \otimes \mathbb{P}^m) \simeq K(\mathbb{P}^n) \otimes K(\mathbb{P}^m) \simeq \mathcal{O}(-(n+m+2)).$

\subparagraph{2.4.7} Am I missing something here? This should just be all collections of $n+m-2$ polynomials such that the intersection is nonsingular and the total degree is $n+m+2$.

\subparagraph{2.4.8} \begin{enumerate}
\item 
\end{enumerate} 
% This follows from the Dolbeault isomorphism 
%\[ H^{0}(M,\Omega^p) \simeq H^{p,0}_{\bar{\partial}}(M),\]
%as well as the fact that the hodge star gives an isomorphism 
%\[ H^{p,0}_{\bar{\partial}}(M) \simeq H^{n-p,n}_{\bar{\partial}}(M) \simeq 0,\]
%since all $
%
%%The $\bar{\partial}$ Poincare lemma gives the exact sequence for each $q > 0$. 
%%\[ 0 \rightarrow \Omega_{\mathbb{P}^n}^{q-1} \rightarrow \mathcal{A}^{q-1} \rightarrow \Omega_{\mathbb{P}^n}^{q-1} \rightarrow 0.\]
%Here $\Omega_{\mathbb{P}^n}^{0} \simeq \mathbb{C}$ and $\mathcal{A}^0 = C^{\infty}(\mathbb{P}^n)$. 
%Now the first map is the inclusion and the second map is $\bar{\partial}$. Now for $q > 0$, $\mathcal{A}^{q}$ is a fine sheaf, so by the long exact sequence in cohomology we have that 
%\[ H^{n}(\mathbb{P}^n,\Omega_{\mathbb{P}^n}^{q-1}) \simeq H^{n-1}(\mathbb{P}^n,\Omega_{\mathbb{P}^n}^{q}).\]
\end{document}