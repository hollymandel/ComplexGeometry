\documentclass[10pt,letter]{article}
\usepackage{amsmath,amssymb,graphicx,setspace,fullpage,breqn,mathtools,stmaryrd,tikz-cd}
\onehalfspacing
\usepackage{fullpage}
\DeclareMathOperator*{\argmin}{arg\,min}

\begin{document}
\noindent 
\begin{center}
\textbf{Some Picard Groups} \\
\line(1,0){250}
\end{center}
\begin{enumerate}
\item {$\mathbb{P}^n$}: Recall that $\text{Pic}(X) \simeq H^{1}(X,\mathcal{O}^{\ast})$.  We know from algebraic topology that $H^k(\mathbb{P}^n,M) = M$ for $k$ even and $0$ otherwise, $M \in \lbrace \mathbb{Z}, \mathbb{Q}, \mathbb{C} \rbrace$.  Therefore the exponential sheaf sequence
\[ H^{1}(\mathbb{P}^n,\mathcal{O}) \rightarrow H^{1}(\mathbb{P}^n,\mathcal{O}^{\ast}) \rightarrow H^2(\mathbb{P}^n,\mathbb{Z})\] shows that $\text{Pic}(X) = \mathbb{Z}$ and that projective line bundles are determined by their image in $H^2(\mathbb{P}^n,\mathbb{Z})$, which is the first Chern class. 
\item {\bf Hypersurfaces in $\mathbb{P}^n$, $n \geq 4$}: . By Chow's theorem, a projective hypersurface is cut out by a section of $\mathcal{O}(d)$ for $d > 0$. Therefore we can apply the Lefschetz hyperplane theorem (GH pg 156) to find that \[ H^1(X,\mathcal{O}) \simeq H^1(X,\mathbb{C}) \simeq H^1(\mathbb{P}^n,\mathcal{O}) = 0,\]
and (because $\text{dim}(X) \geq 3$),
 \[ H^2(X,\mathcal{O}) \simeq H^2(X,\mathbb{C}) \simeq H^2(\mathbb{P}^n,\mathcal{O}) = 0.\]
Therefore using the exponential sequence over $X$ we have that $\text{Pic}(X) \simeq H^2(X,\mathbb{Z})$. But the Lefschetz hyperplane theorem, which is true for integer cohomology, gives that $\mathbb{Z} \simeq  H^2(\mathbb{P}^n,\mathbb{Z}) \simeq H^2(X,\mathbb{Z})$. Therefore $\text{Pic}(X)  \simeq \mathbb{Z}$.
\item {\bf Complete intersection of $k$ hypersurfaces in $\mathbb{P}^n$, $n \geq 3+k$}: We claim that $\text{Pic}(X) \simeq \mathbb{Z}$. The proof proceeds by induction with the first case being the previous bullet. Let $F_k$ be the polynomial of degree $d_k$ defining $S_k$, the $k$th hypersurface. Let $X^{j} = S_1 \cap ... \cap S_{j}$. The requirement that the intersection be complete guarantees that $X^k$ is a smooth hypersurface of $X^{k-1}$. Therefore by Lefschetz and an inductive hypothesis $H^i(X^k,\mathcal{O}) \simeq H^i(X^{k-1},\mathcal{O}) \simeq 0$ for $i = 1,2$ and $H^2(X^{k-1},\mathbb{Z}) \simeq \mathbb{Z}$, so be the exponential sequence and Lefschetz again, $H^2(X^{k},\mathbb{Z}) \simeq \mathbb{Z}$ and $\text{Pic}(X) \simeq H^2(X^{k},\mathbb{Z})$. 
%THIS IS WRONG
%On the other hand, by the naturality of the boundary operator, the following diagram (GH pg 139) commutes:
%\begin{center}
%\begin{tikzcd}
%H^1(\mathbb{P}^n,\mathcal{O}^{\ast}) \arrow[r] \arrow[d, "i^{\ast}"]
%& H^2(\mathbb{P}^n,\mathbb{Z}) \arrow[d, "i^{\ast}"] \\
%H^1(X,\mathcal{O}^{\ast}) \arrow[r]
%& H^2(X,\mathbb{Z})
%\end{tikzcd}
%\end{center}
%The map $H^1(\mathbb{P}^n,\mathcal{O}^{\ast}) \rightarrow H^2(\mathbb{P}^n,\mathbb{Z})$ is injective because $H^1(\mathbb{P}^n,\mathcal{O}) = 0$, while the pullback of Cech cohomology is easily seen to be injective. Therefore $\text{Pic}(\mathbb{P}^n) \hookrightarrow \text{Pic}(X)$. This proves that $\text{Pic}(X) \simeq \mathbb{Z}$. 
\item $H^2(X,\mathcal{O}) = 0$ if the hypersurface is dimension 2? 
\item Curves - especially elliptic ones! This seems cool.

\end{enumerate} 

\end{document}